  %%!TEX root = ./UserManual.tex
\chapter{Introduction}
\label{chap:Introduction}
  
This manual provides a brief description of the Stride software and its
features. 
Stride stands for \textbf{S}imulation of \textbf{tr}ansmission 
 of \textbf{i}nfectious \textbf{d}is\textbf{e}ases and is an agent-based modeling 
system for close-contact disease transmission developed by researchers at the
University of Antwerp and Hasselt University, Belgium.
The simulator uses census-based synthetic populations
that capture the demographic and geographic distributions, as well as detailed social networks.
Stride is an open source software. The authors hope to make large-scale
agent-based epidemic models more useful to the community.
More info on the project and results obtained with the software
can be found in: 
\textit{``Willem L, Stijven S, Tijskens E, Beutels P, Hens N \& Broeckhove J. (2015) Optimizing agent-based transmission models for infectious diseases, BMC Bioinformatics, 16:183''} \cite{willem2015}.

The model population consists of households, schools, workplaces and
communities, which represent a group of people we define as a ``cluster''.
Social contacts can only happen within a cluster. 
When school or work is off, people stay at home and in their primary
community and can have social contacts with the other members.
During other days, people are present in their household, secundary community
and a possible workplace or school.  



We use a $Simulator$ class to organize the activities from the people in an $Area$. 
The Area class has a $Population$, different $Cluster$ objects and a $Contact
Handler$.
The $Contact Handler$ performs Bernoulli trials to decide whether a contact
between an infectious and susceptible person leads to disease transmission. 
People transit through Susceptible-Exposed-Infected-Recovered states,
similar to an influenza-like disease.
Each $Cluster$ contains a link to its members and the $Population$ stores all personal
data, with $Person$ objects.
The implementation is based on the open source model from Grefenstette et al. \cite{grefenstette2013}. 
The household, workplace and school clusters are handled separately from the
community clusters, which are used to model general community contacts. The
$Population$ is a collection of $Person$ objects.

