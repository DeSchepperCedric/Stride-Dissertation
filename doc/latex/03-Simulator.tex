%%!TEX root = ./UserManual.tex
\chapter{Simulator}
\label{chap:simulator}


%%%%%%%%%%%%%%%%%%%%%%%%%%%%%%%%%%%%%%%%%%%%%%%%%%%%%%%%%%%%%%%%
% Workspace
%%%%%%%%%%%%%%%%%%%%%%%%%%%%%%%%%%%%%%%%%%%%%%%%%%%%%%%%%%%%%%%%
\section{Workspace}

By default, \texttt{Stride} is installed in \texttt{./target/installed/} inside de project directory. This can be modified by setting the \texttt{CMAKE\_INSTALL\_PREFIX} on the CMake command line (see the \texttt{INSTALL.txt} file in the prject root directory) or by using the CMakeLocalConfig.txt file (example file can be found  in \texttt{./src/main/resources/make}).

Compilation and installation of the software creates the following files and directories:
\begin{compactitem}
%
	\item Binaries in directory \texttt{<install\_dir>/bin}
      		\begin{compactitem}
        			\item $stride$: executable.
        			\item $gtester$: regression tests for the sequential code.
               	 \item $gengeopop$: generates the population and geographical grid.
                	\item $calibration$: tool to run the simulator multiple times to generate statistical data. 
        			\item $wrapper\_sim.py$: Python simulation wrapper
        	\end{compactitem}
%
    \item Configuration files (xml and json) in directory \texttt{<install\_dir>/config}
      	\begin{compactitem}
			\item $run\_default.xml$: default configuration file for Stride to perform a Nassau simulation.
        		\item $run\_generate\_default.xml$: default configuration file for Stride to first generate a population and geographical grid and then perform a Nassau Simulation.
       		\item $run\_import\_default.xml$: default configuration file for Stride to first import a population and geographical grid and then perform a Nassau Simulation.
        		\item $run\_miami\_weekend.xml$: configuration file for Stride to perform Miami simulations with uniform social contact rates in the community clusters.
			\item $wrapper\_miami.json$: default configuration file for the wrapper\_sim binary to perform Miami simulations with different attack rates.
        		\item \ldots
        \end{compactitem}
%        
    \item Data files (csv) in directory \texttt{<project\_dir>/data}
      	\begin{compactitem}
        		\item $belgium\_commuting$: Belgian commuting data for the active populations. The fraction of residents from ``city\_depart'' that are employed in ``city\_arrival''. Details are provided for all cities and for 13 major cities.
			\item $belgium\_population$: Relative Belgian population per city. Details are provided for all cities and for 13 major cities.
        	\item $flanders\_cities$: Cities and municipalities in Flanders with coordinates and population figures based on samples. These relative population figures are used for assigning residencies and domiciles based on a discrete probability distribution.
        	\item $flanders\_commuting$: Relative commuting information between cities and communities. Since this data is relative, the total number of commuters is a derived parameter, based on the fraction of the total population that is commuting.
			\item $contact\_matrix\_average$: Social contact rates, given the cluster type. Community clusters have average (week/weekend) rates.
			\item $contact\_matrix\_week$: Social contact rates, given the cluster type. Community clusters have week rates.
			\item $contact\_matrix\_week$: Social contact rates, given the cluster type. Primary Community cluster has weekend rates, Secondary Community has week rates.
			\item $disease\_xxx$: Disease characteristics (incubation and infectious period) for xxx.
			\item $holidays\_xxx$: Holiday characteristics for xxx.
			\item $pop\_xxx$: Synthetic population data extracted from the 2010 U.S. Synthetic Population Database (Version 1) from RTI International for xxx \cite{wheaton2014a,wheaton2014b}.
			\item $ref\_2011$: Reference data from EUROSTAT on the Belgian population of 2011. Population ages and household sizes.
			\item $ref\_fl2010\_xxx$: Reference data on social contacts for Belgium, 2011.
        \end{compactitem}
%
    \item Documentation files in directory \texttt{./target/installed/doc}
      	\begin{compactitem}
        			\item Reference manual
        			\item User manual
        \end{compactitem}
%
\end{compactitem}

The install directory is also the workspace for \texttt{Stride}. The \texttt{Stride} executable allows you to use a different output directory for each new calculation (see the next section).

%%%%%%%%%%%%%%%%%%%%%%%%%%%%%%%%%%%%%%%%%%%%%%%%%%%%%%%%%%%%%%%%
% Run
%%%%%%%%%%%%%%%%%%%%%%%%%%%%%%%%%%%%%%%%%%%%%%%%%%%%%%%%%%%%%%%%
\section{Running the simulator}


From the workspace directory, the simulator can be started with default configuration using the command \mbox{``\texttt{./bin/stride}''}. Settings can be passed to the simulator using one or more command line arguments:

\begin{compactitem}	
	\item \texttt{-c} or \texttt{{-}-config}: The configuration file.
\end{compactitem}


%%%%%%%%%%%%%%%%%%%%%%%%%%%%%%%%%%%%%%%%%%%%%%%%%%%%%%%%%%%%%%%%
% genpop
%%%%%%%%%%%%%%%%%%%%%%%%%%%%%%%%%%%%%%%%%%%%%%%%%%%%%%%%%%%%%%%%
\section{Generating a population and geographical grid}

From the workspace directory, the generation of a population and geographical grid (sometimes called GeoGrid) can be started with the default configuration using the command \mbox{``\texttt{./bin/gengeopop}''}. The following configuration options are available:

\begin{compactdesc}
    \item[\texttt{--populationSize}] \ \\
        The size of the population to generate. By default a population of \texttt{6000000} is generated.
    \item[\texttt{--fracActive}] \ \\
        The fraction of people who are active, i.e. who are employed or students. By default \texttt{0.75} is used.
    \item[\texttt{--fracStudentCommuting}] \ \\
        The fraction of students commuting. By default \texttt{0.5} is used.
    \item[\texttt{--fracActiveCommuting}] \ \\
        The fraction of active people who commutes. By default \texttt{0.5} is used.
    \item[\texttt{--frac1826Students}] \ \\
        The fraction of 1826 years which are students. By default \texttt{0.5} is used.
    \item[\texttt{--household}] \ \\
        The file to read the household profiles from.
    \item[\texttt{--commuting}] \ \\
        The file to read the commuting information from.
    \item[\texttt{--cities}] \ \\
        The file to read the cities from.
    \item[\texttt{--output}] \ \\
        The file to write the GeoGrid to. By default this is \texttt{gengeopop.proto}
    \item[\texttt{--state}] \ \\
        The state to be used for initializing the random engine. 
    \item[\texttt{--seed}] \ \\
        The seed to be used for the random engine. The default is \texttt{"1,2,3,4"}.
    \item[\texttt{--loglevel}] \ \\
        The loglevel to use, by default this is \texttt{info}.
\end{compactdesc}

\section{Using the Calibrator}
\label{sec:calibrator}

The Calibrator is a tool designed to calibrate the scenario tests.
It can also be used for running a simulation multiple times to gather statistical data.
This data can then be written to a file or be used for generating boxplots.
The following configuration options are available:

\begin{compactdesc}
    \item[\texttt{--config}] \ \\
        Specifies the run configuration parameters to be used for the simulation.
        If this is provided multiple times, the calibration is performed on all given simulations.
        It may be either \texttt{-c file=<file>} \texttt{or -c name=<name>}.
        The first option can be shortened to \texttt{-c <file>}, the second option accepts \texttt{TestsInfluenza}, \texttt{TestsMeasles} or \texttt{BenchMeasles} as  \texttt{<name}.
    \item[\texttt{--testcases}] \ \\
        Instead of providing the configuration files, you can also select multiple testcases to use for the simulation runs.
        The default is \texttt{influenza\_a, influenza\_b, influenza\_c, measles\_16} and \texttt{r0\_12}.
    \item[\texttt{--multiple}]\ \\
        The amount of simulations to run for each testcase. For each simulation, a different seed will be used.
    \item[\texttt{--single}]\ \\
        Run the simulations with the fixed seeds to determine the exact values.
    \item[\texttt{--output}]\ \\
        Write the results of the calibration to a file with given filename.
        This resulting file contains for each configuration and each step in the simulation the mean, standard deviation, exact value using the default seed and the values found with other seeds.
        These values depend on the selected options, specifically \texttt{--multiple} and \texttt{--single}.
    \item[\texttt{--write}]\ \\
        Write boxplots to files in the current directory. This creates an image for each config or testcase.
    \item[\texttt{--display}]\ \\
        Display the boxplots for the last step.
    \item[\texttt{--displayStep}]\ \\
        Display the boxplots for a specified step.
\end{compactdesc}

Examples:\\
To find the exact values for the testcases and write these to a file:\\

\centerline{\texttt{calibration -s -o out.json}}

To run a configuration file 10 times with a random seed and display the generated boxplot for the last step in the simulation:\\

\centerline{\texttt{calibration -c run\_default.xml -m 10 -d}}

To run the testcase \texttt{influenza\_a} 10 times, write the results to a file and for each step in the simulation write a boxplot to a file:\\

\centerline{\texttt{calibration -t influenza\_a -m 10 -w -o out.json}}



%%%%%%%%%%%%%%%%%%%%%%%%%%%%%%%%%%%%%%%%%%%%%%%%%%%%%%%%%%%%%%%%
% Wrapper
%%%%%%%%%%%%%%%%%%%%%%%%%%%%%%%%%%%%%%%%%%%%%%%%%%%%%%%%%%%%%%%%
\section{Python Wrapper}
A Python wrapper is provided to perform multiple runs with the C++ executable.
The wrapper is designed to be used with .json configuration files and examples are provided with the source code.
For example: \\ \\
\centerline{\texttt{./bin/wrapper\_sim --config ./config/wrapper\_default.json}} \\ \\
will start the simulator with each configuration in the file.
It is important to note the input notation: values given inside brackets can be extended (e.g., ``rng\_seeds''=[1,2,3]) but single values can only be replaced by one other value (e.g., ``days'': 100).


