%%!TEX root = ./UserManual.tex
\chapter{Concepts and Algorithms}
\label{chap:code}


\section{Introduction}
\label{section:ConceptsIntro}

The model population consists of households, schools, workplaces and
communities, which represent a group of people we define as a ``ContactPool''.
Social contacts can only happen within a ContactPool.
When school or work is off, people stay at home and in their primary
community and can have social contacts with the other members.
During other days, people are present in their household, secondary community
and a possible workplace or school.  



We use a $Simulator$ class to organise the activities from the people in the population.
The ContactPools in a population are grouped into ContactCenters (e.g. the different classes of a school are grouped into one K12School ContactCenter).
These ContactCenters are geographically grouped into a geographical grid (sometimes called GeoGrid)

The $Contact Handler$ performs Bernoulli trials to decide whether a contact
between an infectious and susceptible person leads to disease transmission. 
People transit through Susceptible-Exposed-Infected-Recovered states,
similar to an influenza-like disease.
Each $ContactPool$ contains a link to its members and the $Population$ stores all personal
data, with $Person$ objects.
The implementation is based on the open source model from Grefenstette et al. \cite{grefenstette2013}. 
The household, workplace and school clusters are handled separately from the
community clusters, which are used to model general community contacts. The
$Population$ is a collection of $Person$ objects.


%%%%%%%%%%%%%%%%%%%%%%%%%%%%%%%%%%%%%%%%%%%%%%%%%%%%%%%%%%%%%%%%
% GenGeoPop
%%%%%%%%%%%%%%%%%%%%%%%%%%%%%%%%%%%%%%%%%%%%%%%%%%%%%%%%%%%%%%%%

\section{Population on a geographic grid}
\label{section:gengeopop}

\subsection{Background}
\label{subsection:background}

To explain the algorithms used for generating the geography of the countries and their respective population, we have to introduce some background concepts. 

\begin{description}
    \item[ContactPool]:
        A pool of persons that may contact with each other which in turn may lead to disease transmission.
        We distinguish different a number of types of ContactPools associated with the household, the workplace, 
        the school, \ldots. The Household is a key type because it fixed the home address of a person.
    \item[ContactCenter]:
        A group of one or more ContactPools of the same type, at the same Location.
        This construct allows e.g for a K12School (ContactCenter) to contain multiple classes (ContactCenter). 
        The former has a reference size of 500 (again a configurable reference number) pupils in 25 
        (again a configurable reference number) classes.
        A ContactCenter may however also associated one-on-one with a ContactPool; this is the case for 
        ContactCenters of type  Household.
\item[Age Brackets]: 
    		We define age brackets that classify individuals in terms of the type of ContactPool they 
    		can be a member of. Everyone is member of a one Household ContactCenter and ContactPool. For $0 \leqslant age < 3$ one is a possible Daycare student, for $3 \leqslant age < 6$ one is a possible Preschool student, for  $6 \leqslant age < 18$ one is a K12School student, for $18 \leqslant age < 26$ one is either a member of a College
    		ContactCenter or member of a Workplace ContactCenter or neither (i.e. not a college student and not
    		employed). Details of all the rules can be found in the code documentation.
    \item[Daycare student]: 
    		Persons from 0 until 3 years of age that have chosen to attend a Daycare. 
    \item[Preschool student]: 
    		Persons from 0 until 3 years of age that have chosen to attend a Preschool. 		
    \item[K-12 student]: 
    		Persons from 3 until 18 years of age that are required (at least in Belgium) to attend school. 
    		Students that skip or repeat years are not accounted for.
    \item[College student]:
        Persons older than 18 and younger than 26 years of age that attend an institution of higher education. 
        For simplicity we group all forms of higher education into the same type of ContactCenter, a College. 
        A fraction of college students will attend a college ``close to home'' and the others will attend a 
        college ``far from home''. Most higher educations don't last 8 years, but this way we compensate 
        for changes in the field of study, doctoral studies, advanced masters and repeating a failed year of study.
    \item[Employable]:
        We consider people of ages 18 to 65 to be potentially employable. A fraction of people between 18 and 26 
        will attend a college, and the complementary fraction will be employable.
    \item[Active population]:
        The fraction of the employable population that is actually working. A fraction of these workers will 
        work ``close to home'' and the complementary fraction will commute to a workplace ``far from home''.
    \item[Household profile]:
        The composition of households in terms of the number of members and their age is an important 
        factor in the simulation. In this case the profile is not defined by the age of its members or fractions, 
        but through a set of reference households. This set contains a sample of households which is 
        representative of the whole population in their composition.
    \item[GeoGrid locations]:
        Our data only allows for a limited geographical resolution. We have the longitude and latitude of cities 
        and municipalities (a distinction we will not make), which we shall use to create GeoGrid locations for 
        the domicile of the households. All households in the same location are mapped to the coordinates 
        of the location's center. These locations with corresponding coordinates will form a grid that covers 
        the simulation area.
\end{description}

\subsection{Generating the geography}
\label{subsection:gengeo}
We start by generating the geographical component, a GeoGrid.
It contains locations with an id, name, province, coordinates and a reference population count.

The latter requires some comment. When we build a synthetic population, we start by generating households. That household is the assigned a location by drawing from a discrete distribution of locations with weights proportional to the  relative population count of the location. As a consequence, each location will have a population count that differs stochastically from the reference count, but will be close to it. When many synthetic populations are generated, the average of a location 's population count will tend to its reference count.

Locations contain multiple ContactCenters, like Schools and Households, which in turn contain ContactPools.
This structure is internally generated by several ``Generators" and will afterwards be used by ``Populators" to fill the ContactPools. The different types of ContactCenters are created by a different partial generator for each type and added separately to the GeoGrid. We construct the following types of ContactCenters:

\begin{description}
    \item[Households]:
        The number of households is determined by the average size of a household in the reference 
        profile and the total population count.
        The generated households are then assigned to a location by a draw from a discrete distribution of 
        locations with weights proportional to the  relative population count of the location.
    \item[Daycares]:
        Supervises a reference count of 18 children, with a reference count of 18 children per class,
         corresponding to a single ContactPool per daycare. 
         The total number of daycares in the region is determined by the population count, the fraction 
         of people in the Daycare age bracket and of course the reference daycare size.
        The algorithm for assignment of daycares to locations is similar to that of households.   
    \item[PreSchools]:
        Schools a reference count of 120 students, with a reference count of 20 students per class,
         corresponding to 6 ContactPools per preschool. 
         The total number of preschools in the region is determined by the population count, the fraction 
         of people in the Preschool age bracket and of course the reference preschool size.
        The algorithm for assignment of preschools to locations is similar to that of households.
    \item[K-12 Schools]:
        Schools a reference count of 500 students, with a reference count of 20 students per class,
         corresponding to 25 ContactPools per school. 
         The total number of schools in the region is determined by the population count, the fraction 
         of people in the K-12 school age bracket and of course the reference school size.
        The algorithm for assignment of schools to locations is similar to that of households.
    \item[Colleges]:
        Colleges have a reference count of 3000 students with 150 students per ContactPool.
        Colleges are exclusively assigned to the 10 locations with highest reference population count (cities).
        For these 10 locations we use a discrete distribution with weights proportional to the population count 
        of the city relative to the total population of the 10 cities.
    \item[Workplaces]:
        The algorithm for assignment of workplaces to locations is analogous to that of households, 
        but here we factor in commuting information to determine the actual number of workers at a location.
        That is, the reference count of active persons (i.e. persons assigned to a workplace) is 
        the number of active persons living at a location (i.e. their household is located there)
        plus the active persons commuting out of that location minus the number of active persons
        commuting into that location. 
        The reference count for persons at a workplace is 20 when we are not taking workplace size distribution into account. Otherwise the average size of the workplaces is calculated by summing the weighted average for every one of the k workplaces:
        $$size = \sum_{n=1}^{k} ratio_n * (max_n + min_n) / 2$$.
    \item[Communities]:
        We create both primary and secondary communities. Each has a reference count of 2000 persons. 
        Communities consist of persons from all ages. 
        The assignment to locations is again similar to that of households.
\end{description}

\subsection{Generating the population}
\label{subsection:genpop}
After creating the structures that will allow people to come in contact with each other, we have create the population itself and determine the different ContactPools they will be in. The persons are created based on the Household profiles in the HouseholdPopulator. Similar to the approach for the ContactCenters, we have partial populators that 
each populate a type of ContactPool:

\begin{description}
    \item[Households]:
        To create the actual persons, we randomly draw a household from the list of reference households 
        and use that as a template for the number of household members and their ages.
        We simply do this for each household in the GeoGrid, since we already determined 
        the locations and number of households while generating the geography.
        
    \item[Daycares/PreSchools/K-12 Schools]:
        We start by listing all schools within a 10km radius of the household location. 
        If this list is empty, we double the search radius until it's no longer empty.
        We then randomly select a ContactPool from those in the schools of the list,
       even if this ContactPool now has more students than average.
    \item[Colleges]:
        Students who study ``close to home" are assigned to a college with an algorithm
        similar to the assignment to K-12 schools.
        Students that study ``far from home" we first determine the list of locations 
        where people from this locations commute to. We randomly select one of these locations 
        using a discrete distribution based on the relative commuting information.
        We then randomly choose a ContactPool at a college in this location and assign it to 
        the commuting student.
    \item[Workplaces]:
        We first decide whether the person is active (correct age, fraction of the age bracket that is a student,
        fraction of the age bracket that is active).
        We assign a workplace to an active person that works ``close to home" in an 
        analogous way as the assignment of K-12 schools to students.
        For the commuting workers we use an algorithm analogous to that of the commuting college students. When taking workplace size distribution into account, the assignment of workplaces does not happen uniformly. First every workplace gets assigned a workplace type which adds a min and max value. Then the workplace receives a weight based on the current size compared to the min/max. These weights are then used to discretely assign a workplace to a person.
    \item[Communities]:
        The communities we can choose from at a location is determined in an analogous 
        way to the K-12 schools.
        For primary communities we randomly select, for each person in a household, a ContactPool 
        from the list of Communities within a 10km radius. If the list empty, \ldots see the K-12 algorithm.
        In secondary communities however, we assign complete households to the 
        ContactPools instead of each person in the household separately
\end{description}

